\documentclass{article} 
\title{MLang: Music Language} 
\author{Aditya Mukerjee, Nikhil Sarda, Sushmita Swaminathan, Nishtha Agarwal} 
\date{Due: Feb. 20, 2012} 
 
\usepackage{amsmath} 
\usepackage{amsthm} 
 
 
\begin{document} 
\maketitle 


Your symphony is Turing complete. Your composition language should be too.


\section{Introduction}


\emph{‘I can’t define good music, but I know it when I see it!’}\newline



Aesthetics are by nature subjective and non-deterministic; as such, a programming language cannot this problem. But aesthetics are often determined by adherence to well-defined artistic conventions. Just as a simple Venician sonnet must have a fixed meter and rhyming scheme, musical compositions such as fugues, canons, and concertos must obey a well-defined structure specific to that musical form. When composing classical music, a deep understanding of music theory and strong grasp of music notation and grammar are required simply to create a piece compliant with the accepted contemporary musical structure - and that is all before beginning any part of the creative, artistic process. 

However, while the underlying theory is vast, these conventions are extremely well-defined. As a result, most of this process can be facilitated by use of a piece of software specific enough to be tailored to the individual piece of music at hand, yet powerful enough to allow the composer to break free of conventions when necessary, or even to redefine or extend the set of rules. Traditionally, this software is implemented as a graphical frontend to standard music notation. The creators of MLang believe that undermines the power of music. Music is itself a language, and a composition is simply an application of the language, just as a program is an application of a programming language.


The language we have described is simply the language of music - simple and minimal, yet powerful. The same is true of homoiconic functional programming languages, which made the selection of a programming paradigm MLang rather clear. Homoiconic functional languages are simple, yet highly powerful, and they derive this power through their modularity.




\section{Modularity and Metaprogramming}


Music itself is highly modular. Not only is it common for sections of music to be repeated exactly within a given piece of music, but it is often expected that variations on these sections be repeated within the same composition. Subsequently, these sections may included in later compositions by other authors. All music is thus, in some sense, derivative - it is the nature of the art form.  


MLang exposes this modularity, intrinsic in music, and turns it from a barrier to entry into a tool for the beginning composer (and instructors of music composition). Like standard music composition software, MLang allows composers to create music from scratch, by defining notes individually.  Thus, a user familiar with traditional music notation would have little trouble adjusting to MLang. Most music composition software provides this feature and little else. MLang, however, takes this a step further, by allowing the composer to extract arbitrary elements from that music and bundle these elements together into a module.




MLang will allow the user to incorporate other pieces of MLang code - which are playable music files - into their own code as modules. A module is not limited to being just a complete musical phrase, such as a measure, a note, or an entire movement. It may be an attribute of the music, such as the key, time signature, or timbre - these attributes are not necessarily observed (ie, heard) in just a single section of the composition, but rather, they are woven into the composition as a whole. 




\section{Use cases}



Once defined, modules may be transformed in an arbitrary fashion. They may be distributed to another composer who is similarly empowered to transform or modify them before including them in his/her music. Thus, modules are useful to a pair or small group of composers working together. They are also useful for instructors, who may define an arbitrary set of requirements for a composition assignment and provide these to the students (eg, a module that, when included in the composition, fixes the music to a certain key and tempo, and provides a bass line that must be repeated throughout the composition). Or, they may provide a larger module which serves as a template for a less-experienced student. The novice composer can therefore adopt a predefined template for a complicated musical structure, such as a canon, and compose a piece compliant with these standards without the need to fully understand the underlying music theory before beginning.


This is possible because of the homoiconic symmetry between data (music) and MLang code, discussed futher on. Thus, an instructor can provide his or her students with a series of precomposed modules, such as a bass line, a refrain, a coda, or even an entire movement within a symphony. This module can be included in the student’s composition natively, as if he or she had written it. Not only can it be played within the student’s composition, but it can be modified by the student if desired, due to the ease of metaprogramming homoiconic languages. For example an instructor can provide a basso ostinato (short, repeated bass line) that is intended to last the duration of a song. The student may include that as part of their composition as easily as if they had written that component themselves. The student can also make arbitrary adjustments to this component that is given to them before playing it, such as changing the key signature, extending it, contracting it, or more. These changes are represented as functional transformations of the module, so they do not change the state of the module itself, though they do change the end sound.


Thus, an instructor can compose arbitrary components of a musical composition as a framework, and instruct the student to compose the rest (either with or without modifying the precomposed components) as part of an instructive assignment.


\subsection{Example Use cases}

\subsubsection{Use Case 1}
Programmers with basic music knowledge can produce grammatically sound music
The principal purpose of MLang is to introduce to programmers a means of composing theoretically correct music in a familiar setup


1. Programmer specifies desired notes/modules through MLang syntax
2. System checks for syntactical correctness and converts code to corresponding midi file
3. Programmer plays .midi file to listen to composition








\subsubsection{Use Case 2}
 Easy syntax that is simple to learn and intuitive to use
By using the concepts of homoiconicity, MLang aims to keep the code as simple as possible, thus making the task of learning the syntax less daunting. MLang syntax is easier than classical music notation, though it would be straightforward to write a WYSIWYG editor for MLang music as it appears on sheet music, if someone desired. The designers of MLang wish to keep MLang itself simple, however, so this is not part of the language standard.




\section{Homiconicity}

A language is said to be homoiconic if a representation of programs written in it are also a data structure in a primitive type of the language itself. In MLang, the only primitive is a series (list, these lists may be of length one) of musical tones (a note), or a function. However, because functions are themselves stored internally as series of certain musical tones, everything in MLang, including the code itself, is a representation of a series (list) of notes. Modules, the higher-level data structures, are themselves recursively composed of other modules or of primitives (notes). So the code itself has a recursive representation as a primitive datatype. For MLang, the elementary datatype is an m-expression. We define an M-expression inductively as 
1. a musical note, or
2. an expression of the form (x . y) where x and y are m-expressions.
Note the analogous relationship between an m-expression and an s-expression.




\section{Functionality}

MLang is stateless and free of side effects. It treats computation (generation of music) as simple evaluation of m-expressions and avoids state and mutable data. MLang is heavily influenced by the syntax and semantics of Lisp and its variants, though the interpreter itself is written in OCaml. MLang is strongly typed and dynamic.




\section{Primitive data types}


\subsection{Note}
A note has a specified pitch (wave frequency) and duration (temporal length).
The duration is specified as an exact time, in seconds (or part thereof). Changes to the tempo could be represented as functional transformations of the note duration. 


For example, assume an arbitrary time unit. If a particular note has a duration of 5 time-units, by default a 'play' function will play the note for 5 time-units. If the tempo is doubled, a 'play' function will only play that note for 2.5 time-units. In this way, notes themselves are blissfully unaware of the current state of the music, which keeps the code functional. The tempo is a parameter to the 'play' function, and in this way the same note can be shared by many separate calls to the 'play' function, since the note itself is never modified. Thus, the note itself can be treated as if it were immutable for most use cases (the only exception being advanced, deep metaprogramming, as allowed by many popular scripting languages).


There is no intrinsic difference between the internal representations of a note and a function. Both are simply m-expressions. The interpreter will evaluate all expressions. Notes will evaluate to their pitch-duration, and functions will evaluate to the pitch-duration sequence(s) of the note(s) represented.
Therefore, in MLang, the terms 'note' and 'function' can be used interchangeably. For convenience, however, we will use the term 'note' to refer to a single pitch-duration value and 'function' to refer to a lambda-like function that evaluates to one or more notes (one or more pitch-duration values).


For half- and quarter- notes, we may define a single 'unit' that is, by default, equivalent to 100 ms (for example). That way, a transformation that slows down the music would basically treat 1 duration-unit as 200 ms instead of 100 ms. (The note itself would still be stored as 1 duration-unit -- again, since MLang is functional, duration doesn't affect the state of notes; just how they are interpreted/treated).
Alternatively, this information could be passed as an attribute to the note.


A function is structurally identical to a note. As in many other functional programming languages, functions do not 'return' values; they evaluate to values. (In MLang, they may evaluate to notes, modules, or other functions).


Here is an example of what a complete piece would look like
\begin{equation}\newline
(play (tempo 120)\newline
 (instrument violin \newline
   (treble (meter 6 8) \newline
     (c4 q tenuto) (d4 s) (ef4 e sf) \newline
     (c4 e) (d4 s) (en4 s) (fs4 e (fingering 3))) \newline
   (treble (meter 3 4) \newline
     (c5 e. marcato) (d5 s bartok-pizzicato) (ef5 e) \newline
     (c5 e staccato tenuto) (d5 s down-bow) (en5 s) (fs5 e)))\newline
 (instrument piano\newline
   (bar bass (meter 6 16) \newline
         (c4 e tenuto accent rfz) (d4 s mordent) (en4 s pp) (fs4 e fermata))))\newline
\end{equation}\newline

It is fairly straightforward to make out what this code does. It specifies the instruments that we are going to use, a violin (for treble) and a piano (for bass) as well as the meters we are going to use. The individual notes are then specified. Note the construction of these notes; (X Y Z). Here X represents the note that will be played. If X is c4 it implies the C note of the 4th octave. Y indicates the length of the rest, e denoting a whole note, h a half note and q a quarter note. Z represents the style in which the note is played.\newline


Note that, while the code may resemble the Lisp family, it is distinctly different from any widely-known Lisp variant. Lisp syntax is incredibly minimal, defining no syntax except parenthetical statements as delimiters, which allows Lisp to be extended dynamically with ease. Because MLang is also meant to be modular and dynamically extensible, MLang syntax is inspired by but not equivalent to Lisp syntax, just as Java syntax is inspired by but not equivalent to C syntax.


\section{Interpreted}


MLang has a built-in player that can dynamically convert MLang code into polyphonic music. This makes it extremely portable as it can run on any system to which its interpreter has been ported. Alternatively, MLang can also be compiled to MIDI. This can enable it to run on a wide variety of platforms, including ARM devices and synthesizers. Because of the one-to-one correspondence between MLang code and the MIDI files it generates, we conjecture that it would be possible to write a decompiler that can generate MLang code from an arbitrary MIDI file.

\section{Non-primitive datatypes}


In dynamically extensible languages such as Lisp and MLang, there is little-to-no distinction between built-in datatypes and user-defined dataypes - the two are sometimes separated only for performance reasons. Built-in non-primitive datatypes are equivalent, from the user’s perspective, to the built-in datatypes.


Thus, where possible, the built-in modules and datatypes are implemented in MLang. By bootstrapping the language in this manner, the language itself can be extended easily to fit the user’s needs. This feature is possible due to the homoiconicity of the language. Given its homoiconic nature, it is very easy to define new language constructs that can extend the base language in new and interesting ways, including altering the behavior of the built-in datatypes. 



\subsubsection{Base Modules}

Below are modules that will be defined by the MLang standard, thus ensuring their availability. Both are required by the standard; the only difference is that the extra modules must be enabled on a per-program basis, the way that Java built-in standard libraries must be imported. (This is done not for performance reasons, but to keep the syntax simple and minimize potential for namespace collisions).

\subsubsection{Phrase}
A phrase consists of an arbitrary number of notes/functions with an arbitrary total duration. The total duration of a phrase is not known at compile-time. A phrase may be a useful way to segregate or combine notes (or groups of notes), but it is not required.


A phrase could be represented as
(c4 e. wedge) (d4 s staccato) (ef4 e left-hand-pizzicato)
\subsubsection{Trill}

A trill is a pair of notes that are close to each other and played in rapid succession. (For example, in classical music, which has 12 notes on the chromatic scale, the two notes must be two adjacent notes on the chromatic scale, like an F\# and a G).


A trill could be represented as
((fs4 g4) NIL trill)
\subsubsection{Simple Baroque Basso Ostinato}
A Simple Baroque Basso Ostinato consists of two measures played by a bass (ie, the frequencies will be below a certain value). These two measures are intended to be repeated throughout the duration of the composition.
\subsubsection{Sonata-Allegro Form}
A song in Sonata-Allegro form consists of three portions. The first and third are related, so it is often referred to as A-B-A form.
The first and second performance of the 'A' portion differ in that the second 'A' contains no repeated measures (if any exist), and the final few measures are oftentimes replaced with a slightly modified ending, as they become the end of the entire song.
The A portions are in a major key, whereas the B portion is in a minor key. Typically, the minor key for B will be the corresponding minor key for the major key of A, according to classical music theory (ie, C major -> A minor).
\subsubsection{Rondo (assuming parallelism)}
A rondo consists of multiple (typically four) phrases with equal duration, intended to be played in succession. One instrument/voice begins playing the first phrase. As soon as it finishes the first phrase and starts the second, another instrument being playing the first phrase. This process continues, with each instrument beginning the first phrase as the 'previous' instrument finishes it and goes on to the next. As an instrument finishes the final phrase, it may being the first phrase again. Thus, the entire performance of a rondo may loop an arbitrary number of times.
A simple example of a Rondo is 'Row, Row, Row Your Boat'.
\subsubsection{Canon}A canon is similar to a rondo, except that the number of voices/instruments may be less than the number of phrases.



\subsection{Implementation}


The kernel APIs, compiler front-end and the MIDI back-end of MLang are written in ML. ML (and its variants) is a functional language that is eminently suited for implementing domain specific languages. We have used a dialect of ML with object oriented language features known as OCaml. The compiler front-end will parse the MLang code and create an AST out of it. The AST is then flattened to produce an in memory MIDI structure which can either be played on the fly, or serialized to disk. The homoiconicity of MLang also makes it simple to create a REPL for it.


\subsection{Conclusion}


In this whitepaper, we have presented MLang, a DSL for representing music. We believe that it is a promising new step in the field of artificial creativity.


\subsection{Glossary}

\begin{description}

\item{ARM} A 32 bit RISC instruction set architecture that is widely used in many portable devices around the world.


\item{AST} Abstract syntax tree. The fundamental data structure that is populated by the parser which we need to walk or flatten out in order to produce usable output.


\item{MIDI} Music Instrument Digital Interface. An industry spec for encoding, storing, synchronizing, and transmitting the musical performance and control data of electronic musical instruments (synthesizers, drum machines, computers) and other electronic equipment (MIDI controllers, sound cards, samplers).


\item{REPL} Read Eval Print Loop. A sandbox that can be used by developers to prototype and quickly test code.

\end{description}

\end{document} 
 
